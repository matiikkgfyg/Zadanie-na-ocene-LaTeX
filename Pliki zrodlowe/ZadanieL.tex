\documentclass[12pt,a4paper]{report}

\usepackage[T1]{fontenc}
\usepackage[utf8]{inputenc}

\usepackage{geometry}
\geometry{margin=2.5cm}

\usepackage{graphicx}
\usepackage{caption}
\usepackage{booktabs}

\usepackage{amsmath}
\usepackage{hyperref}
\usepackage{fancyhdr}

\pagestyle{fancy}
\fancyhf{}
\fancyhead[LE,RO]{\small\thepage}
\fancyhead[LO]{\small\leftmark}
\fancyhead[RE,RO]{\small Mateusz Dziedzic}
\setlength{\headheight}{15pt}

\title{NVIDIA we współczesnej informatyce:\\
	grafika, sztuczna inteligencja i zastosowania profesjonalne}
\author{Mateusz Dziedzic}
\date{18 listopada 2025}

\begin{document}
	
	\begin{titlepage}
		\centering
		{\Huge\bfseries NVIDIA we współczesnej informatyce\par}
		\vspace{0.5cm}
		{\Large Grafika, sztuczna inteligencja i zastosowania profesjonalne\par}
		\vspace{3cm}
		{\Large Autor: Mateusz Dziedzic\par}
		\vfill
		{\large Data powstania dokumentu: 18 listopada 2025 r.\par}
	\end{titlepage}
	
	\chapter*{Streszczenie}
	Celem dokumentu jest przedstawienie roli firmy NVIDIA
	we współczesnej informatyce. Omówiono historię firmy, rozwój kart graficznych,
	zastosowania w sztucznej inteligencji oraz w obliczeniach wysokiej wydajności.
	Zaprezentowano przykładowe dane w formie tabeli oraz ilustrację poglądową.
	
	W ostatnim rozdziale przedstawiono krótkie wnioski oraz uzasadniono wybór klasy \texttt{report}.
	\clearpage
	
	\tableofcontents
	\clearpage
	
	\chapter{Wprowadzenie do firmy NVIDIA}
	
	Firma NVIDIA jest jednym z najważniejszych producentów układów graficznych
	na świecie. Została założona w latach 90. XX wieku w Stanach Zjednoczonych,
	a jej głównym celem było dostarczanie wydajnych akceleratorów grafiki
	dla rynku gier komputerowych i multimediów \cite{nvidia-history}.
	
	NVIDIA od początku swojej działalności skupiała się na przetwarzaniu równoległym.
	Przełomowym osiągnięciem było wprowadzenie w 1999 roku pierwszego GPU – GeForce 256.
	Tym samym firma zdefiniowała sprzętową akcelerację grafiki 3D, znacząco przyspieszając generowanie obrazu.
	
	W kolejnych latach NVIDIA przejmowała przedsiębiorstwa technologiczne, m.in. 3dfx Interactive oraz Mellanox,
	rozszerzając kompetencje w dziedzinie infrastruktury sieciowej i obliczeń naukowych. Obecnie firma buduje kompletne ekosystemy programistyczne,
	biblioteki dla AI oraz rozwiązania chmurowe.
	
	\section{Historia i kontekst rynkowy}
	
	Pierwsze produkty NVIDIA były skierowane przede wszystkim do użytkowników domowych. Z czasem firma rozszerzyła działalność na:
	
	\begin{itemize}
		\item rynek profesjonalny,
		\item centra danych i obliczenia wysokiej wydajności,
		\item zastosowania związane z uczeniem maszynowym i sztuczną inteligencją.
	\end{itemize}
	
	Rozwój GPU zbiegł się w czasie z intensywnym rozwojem branży gier komputerowych, co stworzyło sprzyjające warunki do ekspansji produktów takich jak seria GeForce \cite{smith-gpu}.
	
	\section{Etapy rozwoju produktów NVIDIA}
	
	\begin{enumerate}
		\item Układy 3D,
		\item Seria GeForce,
		\item Rozwiązania profesjonalne,
		\item GPU dla centrów danych i AI.
	\end{enumerate}
	
	Firma rozwinęła również własne środowiska deweloperskie:
	
	\begin{itemize}
		\item CUDA – platforma do obliczeń równoległych,
		\item TensorRT – optymalizacja modeli sztucznej inteligencji,
		\item Omniverse – system do symulacji 3D i cyfrowych bliźniaków.
	\end{itemize}
	
	\chapter{Technologie i zastosowania GPU NVIDIA}
	
	\section{Karty graficzne GeForce}
	
	Linia GeForce skierowana jest głównie do graczy i zaawansowanych użytkowników domowych.
	Zapewnia wysoką wydajność w grach komputerowych, a modele RTX obsługują technologię ray tracing,
	co umożliwia realistyczne odwzorowanie światła i cieni.
	
	\subsection{Przykładowa ilustracja}
	
	\begin{figure}[h!]
		\centering
		\includegraphics[width=0.6\textwidth]{/home/matt/Pobrane/nvidiaaa1.jpg}
		\caption{MSI Nvidia Geforce GTX 1050TI.}
		\label{fig:gpu-photo}
	\end{figure}
	
	\clearpage
	
	\section{Zastosowania profesjonalne i obliczenia naukowe}
	
	Obecnie GPU NVIDIA używane są w:
	
	\begin{itemize}
		\item studiach filmowych i renderingu 3D,
		\item laboratoriach naukowych,
		\item centrach danych,
		\item zastosowaniach inżynierskich,
		\item medycynie i analizie obrazów medycznych,
		\item modelowaniu zjawisk fizycznych.
	\end{itemize}
	
	W miarę wzrostu zapotrzebowania na moc obliczeniową, NVIDIA stale zwiększa liczbę rdzeni CUDA i Tensor Core w swoich kartach, co pozwala na wykonywanie bilionów operacji na sekundę.
	
	\chapter{NVIDIA i sztuczna inteligencja}
	
	\section{GPU jako akceleratory sieci neuronowych}
	
	GPU wyróżniają się architekturą do wykonywania tysięcy operacji równolegle.
	Jest to idealne do uczenia sieci neuronowych, ponieważ AI opiera się na operacjach macierzowych i wektorowych.
	
	\begin{equation}
		T = \frac{N \cdot E}{P}
	\end{equation}
	
	Zalety GPU w AI:
	
	\begin{enumerate}
		\item Równoległość obliczeń,
		\item Skrócenie czasu trenowania modeli,
		\item Ekosystem programistyczny CUDA wspierający popularne biblioteki (TensorFlow, PyTorch).
	\end{enumerate}
	
	\section{Uproszczone porównanie wybranych kart}
	
	\begin{table}[h!]
		\centering
		\caption{Przykładowe karty graficzne NVIDIA i ich parametry}
		\label{tab:karty}
		\begin{tabular}{lccc}
			\toprule
			Model karty      & Rok premiery & Pamięć [GB] & Zastosowanie \\
			\midrule
			GeForce GTX 1060 & 2016         & 6           & Gry \\
			GeForce RTX 3080 & 2020         & 10          & Gry / AI \\
			NVIDIA A100      & 2020         & 40          & AI / Serwery \\
			\bottomrule
		\end{tabular}
	\end{table}
	
	\chapter{Kilka uwag ,które warto przeczytać}
	
	Do przygotowania dokumentu użyto klasy \texttt{report}, pozwalającej na tworzenie rozdziałów, podrozdziałów oraz osobnej strony tytułowej co było bardzo potrzebne w tym przypadku.
	
	Zastosowane pakiety:
	
	\begin{itemize}
		\item \texttt{geometry} – modyfikacja marginesów,
		\item \texttt{graphicx} – obsługa ilustracji,
		\item \texttt{booktabs} – estetyczne tabele,
		\item \texttt{hyperref} – aktywne odnośniki,
		\item \texttt{fancyhdr} – nagłówki i stopki.
	\end{itemize}
	
	Dzięki nim dokument jest przejrzysty ,wygląda ładnie i funkcjonalny.
	
	\chapter{Wnioski}
		
	Użyłem klasy \texttt{report} ,ponieważ krótkie przedstawienie firmy Nvidia wymagało przejrzystej struktury, spisu treści. Użycie klasy \texttt{Book} było możliwe lecz dokument ,który jest stosunkowo krótki było by trochę nad wyraz bo klasa \texttt{Book} pasuje dobrze do długich dokumentów, książek czy prac Magisterskich.
	
	\section*{Repozytorium z kodem źródłowym}
	
	\begin{center}
		\href{https://github.com/matiikkgfyg/Zadanie-na-ocene-LaTeX.git}{https://github.com/matiikkgfyg/Zadanie-na-ocene-LaTeX.git}
	\end{center}
	\clearpage
	
	\begin{thebibliography}{9}
		\bibitem{nvidia-history}
		NVIDIA Corporation,
		Company History and Milestones, materiały firmowe, dostęp online.
		
		\bibitem{smith-gpu}
		J. Smith,
		The Evolution of Graphics Processing Units, Journal of Computer Graphics, 2020.
		
		\bibitem{goodfellow-dl}
		I. Goodfellow, Y. Bengio, A. Courville,
		Deep Learning, MIT Press, 2016.
		
		\bibitem{nvidia-a100}
		NVIDIA Corporation,
		NVIDIA A100 Tensor Core GPU Architecture, whitepaper techniczny, 2020.
	
		\bibitem{wikimedia-1050ti}
		Wikimedia Commons,
		„Nvidia GeForce GTX 1050 Ti”, licencja CC BY-SA 4.0, dostęp online: 
		\url{https://www.amazon.de/MSI-GeForce-DL-DVI-D-Afterburner-Grafikkarte/dp/B01N683IAQ}, stan na 3 grudnia 2025.
		
		
		
	
	\end{thebibliography}
	
\end{document}
